\PassOptionsToPackage{usenames,dvipsnames}{xcolor}
\documentclass[10pt,a4paper]{moderncv}


\renewcommand{\familydefault}{\sfdefault}
\usepackage{verbatim}


\moderncvtheme[burgundy]{classic}			% optional argument are 'blue' (default), 'orange', 'red', 'green', 'grey' and 'roman' (for roman fonts, instead of sans serif fonts)

% Paquetes
\usepackage{xcolor}
% List of colors http://en.wikibooks.org/wiki/LaTeX/Colors

% character encoding
\usepackage[utf8]{inputenc}			% replace by the encoding you are using

% adjust the page margins
\usepackage[scale=0.82, top=1.2cm, left=1.5cm, right=1.5cm, bottom=1.2cm]{geometry}
\setlength{\hintscolumnwidth}{1.87cm}		% if you want to change the width of the column with the dates
%\AtBeginDocument{\setlength{\maketitlenamewidth}{15.5cm}}  % only for the classic theme, if you want to change the width of your name placeholder (to leave more space for your address details
\AtBeginDocument{\recomputelengths}                     % required when changes are made to page layout lengths





% Comandos
\newcommand{\alto}[1]{\textcolor{ForestGreen}{#1}}
\newcommand{\medio}[1]{\textcolor{BurntOrange}{#1}}
\newcommand{\bajo}[1]{\textcolor{Mahogany}{#1}}


% personal data
\firstname{Agustin}
\familyname{Rojas}
\title{~ \smallskip}
\address{933 Panama, 1185 CABA}{CABA, Argentina}    % optional, remove the line if not wanted
\mobile{+54 11 1565916166}                    % optional, remove the line if not wanted
\email{rojasagustin90@gmail.com}
%\extrainfo{Born: November $6^{th}$, 1991} % optional, remove the line if not wanted


% to show numerical labels in the bibliography; only useful if you make citations in your resume
%\makeatletter
%\renewcommand*{\bibliographyitemlabel}{\@biblabel{\arabic{enumiv}}}
%\makeatother

\nopagenumbers{}                             % uncomment to suppress automatic page numbering for CVs longer than one page
%----------------------------------------------------------------------------------
%            content
%----------------------------------------------------------------------------------
\begin{document}
\maketitle

\vspace{-2cm}

\section{Education}
\cventry{2008 \-- Presente}{Ingeniería en Informática}{Facultad de Ingeniería, Universidad de Buenos Aires (UBA)}{}{Argentina}{
	Estudiante de ultimo año de la carrera de Ingeniería en Informática. Promedio: 7.07/10
}

\cventry{2010 \-- 2017}{Licenciatura en análisis de sistemas}{Facultad de Ingeniería, Universidad de Buenos Aires (UBA)}{}{Argentina}{
	Licenciado de la carrera Licenciatura en análisis de sistemas. Promedio: 7.07/10
}

\cventry{Dic 2015 \-- Dic 2015}{Proyectos agiles con Scrum}{Kleer}{}{Argentina}{
	Curso de proyectos agiles utilizando la metodología Scrum. Dictado por Juan Garbarini (profesor en UBA).
}


\cventry{Jul 2016 \-- Ago 2016}{Curso de Big Data}{Mulesoft Academy}{}{Argentina}{
	Curso de Big Data usando Apache Spark como herramienta. Dictado por Luis Argerich (profesor en UBA).
}

\cventry{Ago 2016 \-- Sept 2016}{Curso de Machine Learning}{Mulesoft Academy}{}{Argentina}{
	Curso de Machine Learning usando Apache Spark como herramienta. Dictado por Luis Argerich (profesor en UBA).
}

\vspace{-0.2cm}


\section{Projects}


\cvline{Primavera 2013}{\textbf{Clasificador de Textos} {\small(C++)} \newline{}
  $\bullet$ Diseño y desarrollo de un algoritmo con la capacidad de clasificar documentos de texto en grupos generados aplicando Clustering (técnica de aprendizaje no supervisado). \newline{}  
  $\bullet$ Implementación de algoritmo de dos etapas. Primera etapa, elegir un grupo de semillas usando Clustering jerarquico aglomerativo. Segunda etapa, algoritmo K-Means usando las semillas elegidas en el paso anterior. \newline{} 
  $\bullet$ Proyecto en equipo para la materia Organización de Datos (FIUBA). \newline{}
}


\vspace{-0.6cm}

\section{Skills}
\cvline{Idiomas}{	\textbf{Inglés}: Intermedio (escrito y hablado). \newline
}
\cvline{Programación}{\medio{\ Scala}, \medio{Python}, \medio{Java}, \medio{C}, \medio{C\#}, \bajo{C++}, \bajo{Node.js}.} 
\cvline{Tecnologías}{ \medio{\ Linux}, \medio{Git}, \medio{Pandas}, \bajo{Sklearn}. \hfill (\alto{high}, \medio{medium}, \bajo{low})}

\vspace{-0.2cm}


\section{Senior Software Engineer}

\cvline{ Mayo 2017 \-- Presente}{	\textbf{Despegar.com} \newline{}
$\bullet$ Implantación y mantenimiento de aplicaciones dentro del checkout en el ambito de la plataforma Despegar.\newline{}
$\bullet$ Desarrollo de herramienta automatizada para ejecutar regresiones para aplicaciones\newline{}
$\bullet$ Colaborador en capacitaciones de técnologias dentro de la empresa.\newline{}
}



\vspace{-0.2cm}


\section{Colaborador}
\cvline{Agosto 2016 \-- Presente}{	\textbf{Taller de Programación II}, Facultad de Ingeniería, Universidad de Buenos Aires: \newline{} 						$\triangleright$ Preparación y dictado de clases \hspace{0.3cm} $\triangleright$  Corrección de trabajos prácticos \hspace{0.3cm} $\triangleright$ Asistencia a estudiantes durante clases prácticas \newline{}
$\bullet$ El curso esta enfocado en brindar las herramientas necesarias para poder profesionalizar al alumno de la carrera. Para esto se dictan clases con distintas tematicas donde se encuentran:Profiling/debugging,Optimización,Buenas prácticas de programación (código bonito),Métricas,Interfaces REST,Ux,Programación asincrónica,Soft-skills,Flujo de trabajo (git flow),Microservices/serverless,Programación WEB }
			

\vspace{-0.2cm}


\section{Software Engineer}

\cvline{ Noviembre 2016 \-- Mayo 2017}{	\textbf{Mercadolibre.com} \newline{}
$\bullet$ Desarrollo y mantenimiento de aplicaciones para el equipo de billing.\newline{}
$\bullet$ Desarrollo utilizando Grails y manejo de grandes volúmenes de datos con distintos motores de base de datos }

\vspace{-0.2cm}

\section{Full Stack Developer}

\cvline{ Agosto 2014 \-- Agosto 2016}{	\textbf{Epson} \newline{}
$\bullet$  Desarrollo y diseño de API embebida para sistemas POS en lenguaje C.\newline{}
$\bullet$ Desarrollo y diseño de PWA para sistemas POS utilizando Angular 1.5\newline{}
$\bullet$ Comunicación con el cliente y captura de nuevos requisitos. }

\vspace{-0.2cm}

\section{Becario}

\cvline{ Agosto 2013 \-- Agosto 2014}{	\textbf{Facultad de Ingenería - UBA} \newline{}
$\bullet$  Programador PHP en la Facultad de ingeniería de la Universidad de Buenos Aires.\newline{}
$\bullet$ Mantenimiento y evolución de los sistemas de computación utilizados para la gestión interna de la institución }


\vspace{-0.2cm}

\section{Programador Java}

\cvline{ Enero 2010 \-- Enero 2012}{	\textbf{Accenture} \newline{}
$\bullet$  Implementación del sistema CC\&B Oracle para grandes clientes (1ro en Latinoamérica).\newline{}
$\bullet$ Relevamiento de información de requerimientos. Análisis de factibilidad y estimación de implementaciones.\newline{}
$\bullet$ Planificación y administración de recursos utilizando herramienta de gestión MS Project.\newline{}
$\bullet$ Programación de servicios Java J2EE con Hibernate y Oracle PL-SQL bajo plataforma CC\&B.\newline{}
$\bullet$ Soporte técnico a diversos sectores operacionales (administradores de base de datos)\newline{}
$\bullet$ administradores de servidores Unix, Seguridad informática).\newline{}
$\bullet$ Control de la mesa de ayuda, incidencias y consultas de usuario.\newline{}
$\bullet$ Gestión de reportes Bi Publisher.
}

\end{document}
