\PassOptionsToPackage{usenames,dvipsnames}{xcolor}
\documentclass[10pt,a4paper]{moderncv}


\renewcommand{\familydefault}{\sfdefault}
\usepackage{verbatim}


\moderncvtheme[burgundy]{classic}			% optional argument are 'blue' (default), 'orange', 'red', 'green', 'grey' and 'roman' (for roman fonts, instead of sans serif fonts)

% Paquetes
\usepackage{xcolor}
% List of colors http://en.wikibooks.org/wiki/LaTeX/Colors

% character encoding
\usepackage[utf8]{inputenc}			% replace by the encoding you are using

% adjust the page margins
\usepackage[scale=0.82, top=1.2cm, left=1.5cm, right=1.5cm, bottom=1.2cm]{geometry}
\setlength{\hintscolumnwidth}{1.87cm}		% if you want to change the width of the column with the dates
%\AtBeginDocument{\setlength{\maketitlenamewidth}{15.5cm}}  % only for the classic theme, if you want to change the width of your name placeholder (to leave more space for your address details
\AtBeginDocument{\recomputelengths}                     % required when changes are made to page layout lengths





% Comandos
\newcommand{\alto}[1]{\textcolor{ForestGreen}{#1}}
\newcommand{\medio}[1]{\textcolor{BurntOrange}{#1}}
\newcommand{\bajo}[1]{\textcolor{Mahogany}{#1}}


% personal data
\firstname{Fabrizio Sebastian}
\familyname{Graffe}
\title{~ \smallskip}
\address{3162 Carlos Gardel, 1215 CABA}{CABA, Argentina}    % optional, remove the line if not wanted
\phone{+54 11 42928787}
\mobile{+54 11 1551793613}                    % optional, remove the line if not wanted
\email{zebas.graffe@gmail.com}
%\extrainfo{Born: November $6^{th}$, 1991} % optional, remove the line if not wanted


% to show numerical labels in the bibliography; only useful if you make citations in your resume
%\makeatletter
%\renewcommand*{\bibliographyitemlabel}{\@biblabel{\arabic{enumiv}}}
%\makeatother

\nopagenumbers{}                             % uncomment to suppress automatic page numbering for CVs longer than one page
%----------------------------------------------------------------------------------
%            content
%----------------------------------------------------------------------------------
\begin{document}
\maketitle

\vspace{-2cm}

\section{Education}
\cventry{2010 \-- Presente}{Ingeniería en Informática}{Facultad de Ingeniería, Universidad de Buenos Aires (UBA)}{}{Argentina}{
	Estudiante de ultimo año de la carrera de Ingeniería en Informática. Promedio: 7.07/10
}

\cventry{Jul 2016 \-- Ago 2016}{Curso de Big Data}{Mulesoft Academy}{}{Argentina}{
	Curso de Big Data usando Apache Spark como herramienta. Dictado por Luis Argerich (profesor en UBA).
}

\cventry{Ago 2016 \-- Sept 2016}{Curso de Machine Learning}{Mulesoft Academy}{}{Argentina}{
	Curso de Machine Learning usando Apache Spark como herramienta. Dictado por Luis Argerich (profesor en UBA).
}

\vspace{-0.2cm}


\section{Projects}


\cvline{Primavera 2013}{\textbf{Clasificador de Textos} {\small(C++)} \newline{}
  $\bullet$ Diseño y desarrollo de un algoritmo con la capacidad de clasificar documentos de texto en grupos generados aplicando Clustering (técnica de aprendizaje no supervisado). \newline{}  
  $\bullet$ Implementación de algoritmo de dos etapas. Primera etapa, elegir un grupo de semillas usando Clustering jerarquico aglomerativo. Segunda etapa, algoritmo K-Means usando las semillas elegidas en el paso anterior. \newline{} 
  $\bullet$ Proyecto en equipo para la materia Organización de Datos (FIUBA). \newline{}
}


\vspace{-0.6cm}

\section{Skills}
\cvline{Idiomas}{	\textbf{Inglés}: Avanzado (escrito y hablado). \newline  Actualmente asistiendo a clases in-company una vez a la semana.
}
\cvline{Programación}{\medio{\ Scala}, \medio{Python}, \medio{Java}, \medio{C}, \medio{C\#}, \bajo{C++}, \bajo{Node.js}.} 
\cvline{Tecnologías}{ \medio{\ Linux}, \medio{Apache Spark}, \medio{Git}, \medio{Pandas}, \bajo{Sklearn}. \hfill (\alto{high}, \medio{medium}, \bajo{low})}

\vspace{-0.2cm}


\section{Data Engineer}

\cvline{Abr 2018 \-- Presente}{	\textbf{Redbee studios}, Software factory: \newline{}
\textbf{Ofertador:} 	\newline{} 
$\bullet$ Creación de prueba de concepto de motor de recomendaciones de productos financieros usando algoritmos de machine learning desarrollados en Python utilizando la biblioteca scikit-learn. Esto permitió la venta del proyecto a Cencosud. El motor fué portado mas adelante a Spark ML.
$\bullet$ Las tareas involucraron exploración de los datos, investigación acerca de diferentes modelos de ML,  desarrollo de prueba de concepto y luego de eso, un MVP el cual está actualmente desplegado y funcionando en un ambiente productivo.
$\bullet$ Actualmente trabajando para mejorar la calidad de las recomendaciones, optimizar la performance del pipeline de datos y agregar nuevas fuentes de datos.
}

\cvline{ Feb 2017 \-- Abr 2018}{	\textbf{Redbee studios}, Software factory: \newline{}
$\bullet$ Trabajo con equipo remoto situado en Seattle (USA) para empresa con producto de sistema de recomendaciones omnicanal con grandes clientes a nivel mundial. Comunicaciones en inglés. \newline{}
\textbf{Equipo Data, RichRelevance:} 	\newline{} 
$\bullet$ Desarollo de ETLs que procesan comportamiendo de los usuarios en las páginas y producen reportes con insights sobre dichos datos en HDFS. Se usó Apache Spark (Scala) para el procesamiento de grandes volumenes de datos y ThoughtSpot para la creación de visualizaciones.
$\bullet$ Se desarrolló una health check API (Play Framework) para reportar sitios con problemas, utilizando desviaciones standard de la media para calcular la severidad.
$\bullet$ Git Flow como esquema de branchs.
\newline{}
\textbf{Equipo Science Dev, RichRelevance:} 	\newline{}
$\bullet$ Tecnologías usadas: Scala + Apache Spark, Java + Apache Crunch.
$\bullet$ Refactor a modelo de machine learning.
$\bullet$ Desarrollo  de prueba de concepto para implementación alternativa de uno de los productos de recomendación principales de la empresa. 
$\bullet$ Investigación para mejorar el uso de memoria para un algoritmo de predicción de clicks en ads.
}

\vspace{-0.2cm}


\section{Colaborador}
\cvline{ 2012 \-- 2015}{	\textbf{Algoritmos y Programación I}, Facultad de Ingeniería, Universidad de Buenos Aires: \newline{} 						$\triangleright$ Preparación y dictado de clases \hspace{0.3cm} $\triangleright$  Corrección de trabajos prácticos \hspace{0.3cm} $\triangleright$ Asistencia a estudiantes durante clases prácticas \newline{}
$\bullet$ El curso usa python como lenguaje de aprendizaje y cubre temas como estructuras condicionales, ciclos definidos e indefinidos, definición de funciones, estructuras de datos, algoritmos de ordenamiento, recursividad y nociones de programación orientada a objetos. }
			
					
\end{document}
