%%%%%%%%%%%%%%%%%%%%%%%%%%%%%%%%%%%%%%%%%
% Masters/Doctoral Thesis 
% LaTeX Template
% Version 2.5 (27/8/17)
%
% This template was downloaded from:
% http://www.LaTeXTemplates.com
%
% Version 2.x major modifications by:
% Vel (vel@latextemplates.com)
%
% This template is based on a template by:
% Steve Gunn (http://users.ecs.soton.ac.uk/srg/softwaretools/document/templates/)
% Sunil Patel (http://www.sunilpatel.co.uk/thesis-template/)
%
% Template license:
% CC BY-NC-SA 3.0 (http://creativecommons.org/licenses/by-nc-sa/3.0/)
%
%%%%%%%%%%%%%%%%%%%%%%%%%%%%%%%%%%%%%%%%%

%----------------------------------------------------------------------------------------
%	PACKAGES AND OTHER DOCUMENT CONFIGURATIONS
%----------------------------------------------------------------------------------------

\documentclass[
11pt, % The default document font size, options: 10pt, 11pt, 12pt
%oneside, % Two side (alternating margins) for binding by default, uncomment to switch to one side
spanish, % ngerman for German
singlespacing, % Single line spacing, alternatives: onehalfspacing or doublespacing
%draft, % Uncomment to enable draft mode (no pictures, no links, overfull hboxes indicated)
%nolistspacing, % If the document is onehalfspacing or doublespacing, uncomment this to set spacing in lists to single
%liststotoc, % Uncomment to add the list of figures/tables/etc to the table of contents
%toctotoc, % Uncomment to add the main table of contents to the table of contents
%parskip, % Uncomment to add space between paragraphs
%nohyperref, % Uncomment to not load the hyperref package
headsepline, % Uncomment to get a line under the header
%chapterinoneline, % Uncomment to place the chapter title next to the number on one line
%consistentlayout, % Uncomment to change the layout of the declaration, abstract and acknowledgements pages to match the default layout
]{MastersDoctoralThesis} % The class file specifying the document structure


\selectlanguage{spanish}

\usepackage[utf8]{inputenc} % Required for inputting international characters
\usepackage[T1]{fontenc} % Output font encoding for international characters

\usepackage{mathpazo} % Use the Palatino font by default

\usepackage[backend=biber,style=authoryear,natbib=true]{biblatex} % Use the bibtex backend with the authoryear citation style (which resembles APA)

\addbibresource{references.bib} % The filename of the bibliography

\usepackage[autostyle=true]{csquotes} % Required to generate language-dependent quotes in the bibliography

\usepackage{tabularx}


%----------------------------------------------------------------------------------------
%	MARGIN SETTINGS
%----------------------------------------------------------------------------------------

\geometry{
	paper=a4paper, % Change to letterpaper for US letter
	inner=2.5cm, % Inner margin
	outer=3.8cm, % Outer margin
	bindingoffset=.5cm, % Binding offset
	top=1.5cm, % Top margin
	bottom=1.5cm, % Bottom margin
	%showframe, % Uncomment to show how the type block is set on the page
}


%----------------------------------------------------------------------------------------
%	THESIS INFORMATION
%----------------------------------------------------------------------------------------

\thesistitle{DeltaML \\
Machine Learning descentralizado utilizando Blockchain para fomentar participaci\'on en la red} % Your thesis title, this is used in the title and abstract, print it elsewhere with \ttitle
\supervisor{Dr. Mariano \textsc{Beiró}} % Your supervisor's name, this is used in the title page, print it elsewhere with \supname
\examiner{} % Your examiner's name, this is not currently used anywhere in the template, print it elsewhere with \examname
\degree{Ingeniería en Informática} % Your degree name, this is used in the title page and abstract, print it elsewhere with \degreename
\author{
Fabrizio \textsc{Graffe} 93158\\
Agustín \textsc{Rojas}  91462
} % Your name, this is used in the title page and abstract, print it elsewhere with \authorname
\addresses{} % Your address, this is not currently used anywhere in the template, print it elsewhere with \addressname

\subject{Inform'\atica} % Your subject area, this is not currently used anywhere in the template, print it elsewhere with \subjectname
\keywords{} % Keywords for your thesis, this is not currently used anywhere in the template, print it elsewhere with \keywordnames
\university{\href{http://www.uba.ar}{Universidad de Buenos Aires}} % Your university's name and URL, this is used in the title page and abstract, print it elsewhere with \univname
\department{\href{http://www.fi.uba.ar}{Facultad de Ingenier\'ia}} % Your department's name and URL, this is used in the title page and abstract, print it elsewhere with \deptname
\group{\href{http://researchgroup.university.com}{Research Group Name}} % Your research group's name and URL, this is used in the title page, print it elsewhere with \groupname
\faculty{\href{http://www.fi.uba.ar}{Facultad de Ingenier\'ia}} % Your faculty's name and URL, this is used in the title page and abstract, print it elsewhere with \facname

\AtBeginDocument{
\hypersetup{pdftitle=\ttitle} % Set the PDF's title to your title
\hypersetup{pdfauthor=\authorname} % Set the PDF's author to your name
\hypersetup{pdfkeywords=\keywordnames} % Set the PDF's keywords to your keywords
}

\begin{document}

\frontmatter % Use roman page numbering style (i, ii, iii, iv...) for the pre-content pages

\pagestyle{plain} % Default to the plain heading style until the thesis style is called for the body content


%----------------------------------------------------------------------------------------
%	TITLE PAGE
%----------------------------------------------------------------------------------------

\begin{titlepage}
\begin{center}

\vspace*{.06\textheight}
{\scshape\LARGE \univname\par}
\vspace{0.7cm} % University name

\includegraphics[scale=0.3]{Logo} % University/department logo - uncomment to place it

\vspace{0.7cm} % University name


\textsc{\Large Propuesta de Trabajo Profesional}\\[0.5cm] % Thesis type

\HRule \\[0.4cm] % Horizontal line
{\huge \bfseries \ttitle\par}\vspace{0.4cm} % Thesis title
\HRule \\[1.5cm] % Horizontal line
 
\begin{minipage}[t]{0.4\textwidth}
\begin{flushleft} \large
\emph{Author:}\\
\href{http://www.johnsmith.com}{\authorname} % Author name - remove the \href bracket to remove the link
\end{flushleft}
\end{minipage}
\begin{minipage}[t]{0.4\textwidth}
\begin{flushright} \large
\emph{Supervisor:} \\
\href{http://www.jamessmith.com}{\supname} % Supervisor name - remove the \href bracket to remove the link  
\end{flushright}
\end{minipage}\\[3cm]

 
\vfill

{\large \today}\\[4cm] % Date

 
\vfill
\end{center}
\end{titlepage}

%----------------------------------------------------------------------------------------
%	LIST OF CONTENTS/FIGURES/TABLES PAGES
%----------------------------------------------------------------------------------------

\tableofcontents % Prints the main table of contents

%----------------------------------------------------------------------------------------
%	THESIS CONTENT - CHAPTERS
%----------------------------------------------------------------------------------------

\mainmatter % Begin numeric (1,2,3...) page numbering

\pagestyle{thesis} % Return the page headers back to the "thesis" style

\chapter{Introducci\'on}

\chapter{Descripci\'on del problema}

\chapter{Estado del arte}

\chapter{Objetivos}

\chapter{Car\'acteristicas del trabajo}


\section{Modulo de Machine Learning descentralizado}


\section{Modulo de encriptac\'ion}

\section{}


\chapter{Tecnolog\'ias}

\section{Blockchain}

\subsection{Ethereum network}
\parencite{Ethereum}


\subsection{Solidity}

\subsection{Zeppelin OS}

\subsection{Open Zeppelin}

\subsection{Truffle}

\subsection{Ganache}

\subsection{Metamask}


\section{Machine Learning}


\section{Frontend}

\subsection{React}

\section{Storage}

\subsection{IPFS}

\chapter{Alcance}

\chapter{Plan de Trabajo}

\section{Equipo de trabajo}

\begin{center}
	\begin{tabularx}{\textwidth}{|X|X|}
    \hline
	\multicolumn{2}{ |c| }{\textbf{Equipo}} \\    
    \hline
    \textbf{Integrante} & \textbf{Rol}  \\ \hline
    Mariano Beiró & Tutor \\ \hline
    Fabrizio Graffe & Desarrollador \\ \hline
    Agustín Rojas & Desarrollador \\ \hline
    \end{tabularx}
\end{center}


\section{Metodolog\'ia}

Para la ejecución del proyecto se usará una metodología ágil basada en SCRUM. La misma consistirá en definir una serie de iteraciones con fechas pautadas de reuniones de avance y entregas.

Al inicio de cada iteración se conformará una priorización de los requerimientos pendientes de desarrollo. Posteriormente, se procederá a su implementación y, para finalizar cada iteración, se hará una presentación de los entregables pautados. Las reuniones, entregas, presentaciones y la priorización de los requerimientos para cada iteración se realizará en con el tutor del trabajo.



\section{Estimaci\'on}

Se muestra a continuación el listado de tareas necesarias para alcanzar los objetivos descritos anteriormente. Todos los esfuerzos se encuentran expresados en horas.

{
\setlength{\extrarowheight}{.2em}
\begin{center}
	\begin{tabularx}{\textwidth}{|l|X|l|}
    \hline
    \textbf{It.} & \textbf{Descripción} & \textbf{Esf.}  \\ \hline
    1 & Aprendizaje desarrollo de aplicaciones descentralizadas & 40 \\ 
      & Aprendizaje de uso de IPFS como storage descentralizado & 10 \\
      & Desarrollo de prueba de concepto (IPFS + Ethereum) & 20 \\ \hline
    2 & Investigación sobre métodos de encripción & 40 \\
      & Desarrollo de modulo de encriptación de modelos de ML & 20 \\ \hline
    3 & Investigación sobre Federated Learning y alternativas & 40 \\ 
      & Desarrollo de modulo de Federated Learning & 20 \\ 
      & Desarrollo de caso de uso para modulo de Federated Learning & 20 \\ \hline
    4 & Desarrollo de aplicación web para marketplace descentralizado & 20 \\ 
      & Desarrollo de smart contracts para marketlace descentralizado & 20 \\ 
      & Integración de los modulos desarrollados con el marketplace descentralizado & 60 \\ \hline     
    5  & Pruebas del sistema en funcionamiento & 20 \\ \hline
      & Reuniones & 15 \\ \hline
      & Presentación & 15 \\ \hline
      & Reuniones & 15 \\ \hline
    \end{tabularx}
\end{center}
}

Además se debe tener en cuenta el tiempo dedicado a la administración del proyecto, estimando un 15\% del tiempo de desarrollo, lo cual resulta en:

{
\setlength{\extrarowheight}{.2em}
\begin{center}
	\begin{tabular}{|l|l|}
    \hline
    \textbf{Descripción} & \textbf{Esf.}  \\ \hline
    Iteración 1 & 70 \\ \hline
    Iteración 2 & 60 \\ \hline
    Iteración 3 & 80 \\ \hline
    Iteración 4 & 100 \\ \hline
    Iteración 5 & 20 \\ \hline
    Otros & 45 \\ \hline     
    Administración & 105 \\ \hline
    \textbf{Esfuerzo total} & 805 \\ \hline
    \end{tabular}
\end{center}
}

\section{Cronograma de entregables}
A continuación se hace un cronograma tentativo de entregables al finalizar cada una de las iteraciones. Las mismas pueden ser modificadas en caso de verse necesario, por el tutor o por los desarrolladores del proyecto. La fecha de cada una de las entregas serán determinadas en conjunto con el tutor.

{
\setlength{\extrarowheight}{.2em}
\begin{center}
	\begin{tabularx}{\textwidth}{|l|X|X|}
    \hline
    \textbf{It.} & \textbf{Entregables} & \textbf{Hitos}  \\ \hline
    1 & Prueba de concepto (IPFS + Ethereum) & Curso de aplicaciones descentralizadas terminado  \\ 
      & & Videos y lecturas sobre IPFS terminadas \\
      & & Prueba de concepto desarrollada  \\ \hline
    2 & Modulo de encriptación de modelos & Investigación sobre hommomorphic encryption, multy-party encryption y alternativas terminada. \\ 
      & & Método de encriptación seleccionado \\
      & & Modulo de encriptación terminado \\ \hline
    3 & Modulo de Federated Learning & Investigación sobre Federated Learning y alternativas terminado. \\ 
      & & Pruebas de conepto usando Federated Learning terminadas. \\
      & & Desarrollo de framework de ML usando Federated Learning terminado. \\ \hline
    4 & Marketplace descentralizado usando smart contracts sobre Ethereum & Desarrollo de smart contracts terminado. \\
      & & Desarrollo de frontend terminado. \\
      & & Integración de frontend con smart contracts e IPFS terminado. \\
      & & Desarrollo de caso de prueba terminado \\ \hline 
    5 & Casos de uso probados & Prueba exhaustiva de uno o dos casos de uso terminada \\ \hline
    \end{tabularx}
\end{center}
}


%----------------------------------------------------------------------------------------
%	BIBLIOGRAPHY
%----------------------------------------------------------------------------------------

\printbibliography[heading=bibintoc]


%----------------------------------------------------------------------------------------

\end{document}  
