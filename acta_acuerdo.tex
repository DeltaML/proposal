\documentclass[a4paper, 12pt]{iopart}

\usepackage[]{graphicx}
\usepackage[left=2.5cm,top=1.5cm,right=2.5cm,nohead,nofoot]{geometry}

\graphicspath{{fig/}{img/}}

%Acentos y espa�ol
\usepackage[latin1]{inputenc}
\usepackage[spanish]{babel}

\pagestyle{empty}

%%%%%%%%%%%%%%%%%%%%%%%%%%%%%%%%%%%%%%%%%%%%%%%%%%%%%%%%%%%%%%%%%%%
\begin{document}
%%%%%%%%%%%%%%%%%%%%%%%%%%%%%%%%%%%%%%%%%%%%%%%%%%%%%%%%%%%%%%%%%%%

\maketitle 

%%%%%%%%%%%%%%%%%%%%%%%%%%%%%%%%%%%%%%%%%%%%%%%%%%%%%%%%%%%%%%%%%%%%%%%%%%%%% 
\section*{}\label{carta}
%%%%%%%%%%%%%%%%%%%%%%%%%%%%%%%%%%%%%%%%%%%%%%%%%%%%%%%%%%%%%%%%%%%%%%%%%%%%% 

\vspace{0.4cm}

\justify
En la ciudad de Buenos Aires al d�a 19 de febrero de 2019 se re�nen el profesor del Departamento de Computaci�n Dr. Mariano Gast�n Beir� y los estudiantes de la carrera de Ingenier�a en Inform�tica Fabrizio Graffe (padr�n 93.158) y Agust�n Rojas (padr�n 91.462) para acordar su tema de Trabajo Profesional de Ingenier�a Inform�tica.

\justify
Luego de haber conversado sobre las �reas de inter�s de los estudiantes, se acuerda establecer el siguiente proyecto de Trabajo Profesional en Ingenier�a Inform�tica: \textit{``DeltaML: Plataforma descentralizada de machine learning sin violar la privacidad del usuario''}.

\justify
El profesor Mariano Beir� deja constancia de que en su opini�n las asignaturas de Orientaci�n y Electivas realizadas por ambos estudiantes abarcan los conocimientos necesarios para que desarrollen satisfactoriamente su Trabajo Profesional, por lo que se puede dar por cumplida la elaboraci�n del Plan de Estudio Personal.

\justify
Sin m�s que tratar, se da por concluida la reuni�n. \\

\vspace{3.0cm}

--------------------------------- \hspace{0.45cm} --------------------------------- \hspace{0.45cm} -----------------------------------

\hspace{0.45cm} Sr. Fabrizio Graffe \hspace{1.9cm} Sr. Agust�n Rojas \hspace{1.2cm} Dr. Mariano Gast�n Beir�

\end{document}

